\documentclass[a4paper,10pt]{article}
\usepackage{fancyhdr}
\usepackage{graphicx}
\usepackage{geometry}
\usepackage[colorlinks=true, linkcolor=blue]{hyperref}
\usepackage{xcolor}
\usepackage{soul}
% Customize section formatting
\usepackage{titlesec}
\usepackage{enumitem}



\definecolor{customblue}{HTML}{31B6FD}

\titleformat{\section}[hang]{\normalfont\large\bfseries}{\thesection}{1em}{}[\color{customblue}\rule{\linewidth}{1.5pt}]
\titlespacing*{\section}{0pt}{\baselineskip}{\baselineskip}


% Set the page geometry
\geometry{left=0.5in, right=0.5in, top=1.25in, bottom=1in} % Increase top margin

% Set up the fancyhdr package
\pagestyle{fancy}
\fancyhf{} % Clear all header and footer fields

% Define the header
\fancyhead[L]{\includegraphics[height=1.5cm]{./assets/logo.png}} % Adjust height as needed
\fancyhead[C]{\textbf{INSTITUTE FOR\\CO-OPERATIVE EDUCATION}} % Text in bold
\renewcommand{\headrulewidth}{1pt} % Set the thickness of the header rule
\renewcommand{\headrule}{\hbox to\headwidth{\color{customblue}\leaders\hrule height \headrulewidth\hfill}}

% Ensure enough space for the header
\setlength{\headheight}{2.5cm}
\addtolength{\topmargin}{-1.5cm}


% Start of document
\begin{document}

% 
\begin{center} % Center the content
    {\LARGE \textbf{FIRST NAME LAST NAME}} \\
    \href{mailto:professionalemail@gmail.com}{professionalemail@gmail.com} $\mid$ 514.123.4567 \\
    \href{https://www.linkedin.com/in/yourlinkedinprofile}{LinkedIn} $\mid$ \href{https://github.com/yourgithub}{GitHub}
\end{center}

\noindent
Date of application

\vspace{1em}

\noindent
First Name Last Name of contact person \\
\textit{(If you are sending this to HR include that person's name here)} \\
Position/Title (e.g., Hiring Manager/Supervisor) \\
Name of company/organization \\
Company mailing address \\
City, Province Postal Code

\vspace{1em}

\noindent
\textbf{Subject: Name of Position (Reference Number, if provided) with Company} \\
\textit{For example: Re: Java Software Developer with Ericsson Canada Inc. (Offer: 1234)}

\vspace{1em}

\noindent
Dear [Ms. or Mr. Last Name], \\
\textit{(Address this to the person you are responding to or the 
person you will be reporting to. If you cannot find the name 
address it “Dear Hiring Manager,” “Dear Hiring Committee” 
“Dear Human Resource Manager”). Avoid using the generic term “To Whom it May Concern”)}

\vspace{1em}

\noindent
Introduce what position you are responding to and 
why you are interested. You can include your field of 
study and what year you are in if you think this will add value. 
You want to create a rapport and grab the reader’s attention for 
them to want to read on. The first paragraph must grab the reader’s 
attention and concisely show why you are interested and the right person for the position.

\vspace{1em}

\noindent
This body tells the reader why they should select you over the 
other qualified candidates. The goal is to make an explicit link 
between what the employer is looking for and how you are going to help 
them meet that need with the skills and qualities that you can bring 
to the role/position/company. Talk about what you can do for the employer, 
not what the employer can do for you. Select a few broad categories 
that best match what the employer needs and wants. This paragraph should 
be based on the position description in the posting, and how you will be 
able to meet the requirements (backed-up with clear, concise, impactful examples). 
You can add a few things like your willingness to travel, relocate, 
any relevant training or course you are learning that would be important 
for the employer to know.

\vspace{1em}

\noindent
Your final paragraph should express your interest in the company 
and thank the reader appropriately. \textbf{Wrap it up with a call to action.} 
Inform the reader that you look forward to an interview to 
further learn about the opportunity and to discuss your interest in the role.

\vspace{1em}

\noindent
Sincerely,

\vspace{1em}

\noindent
First Name Last Name


\end{document}
